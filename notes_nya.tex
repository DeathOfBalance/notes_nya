\documentclass[12pt]{article}
%\usepackage{francais}
\usepackage{amssymb, latexsym, amsmath,amsthm}
\usepackage{polyglossia}
\setmainlanguage{french} % pour compiler avec xelatex
%\usepackage[francais]{babel}
%\usepackage[utf8x]{inputenc}
%\usepackage[utf8]{inputenc}
\usepackage{tikz}
\usepackage{wrapfig}
\usepackage{geometry}
\geometry{margin=1in}
\usepackage{enumitem}
\renewcommand{\labelenumi}{\alph{enumi})}
\renewcommand{\theenumi}{\alph{enumi})}
\usepackage{multicol}

\usepackage[bottom]{footmisc}
\renewcommand{\thefootnote}{\fnsymbol{footnote}}

\newlist{problemes}{enumerate}{1}
\setlist[problemes]{label=\textbf{Problème \arabic*}, align=left}
\setcounter{section}{-1}

\theoremstyle{definition}
\newtheorem{ex}{Example}

\theoremstyle{remark}
\newtheorem*{rem}{Remarque}

\newcommand{\arccot}{\operatorname{arccot}}
\newcommand{\R}{\mathbb{R}}
\newcommand{\N}{\mathbb{N}}
\newcommand{\C}{\mathbb{C}}

\begin{document}

\title{Notes pour 201-NYA (Calcul différentiel)}
\author{Alex Provost}
\maketitle

\section{Révision}
\section{Limites et continuité}
\section{Dérivée}
\section{Analyse et optimisation}
\section{Fonctions transcendantes}

Jusqu'à présent, nous n'avons considéré que des fonctions \emph{algébriques}, c'est-à-dire des fonctions construites à partir des opérations algébriques usuelles (somme, différence, produit, quotient, racines).
%Par exemple, un polynôme est simplement une somme de produits de constantes et de fonctions identité. %todo : rephraser? exemples?
Il existe cependant beaucoup d'autres fonctions importantes qui ne peuvent \emph{pas} s'exprimer simplement à l'aide des opérations algébriques usuelles. Ces fonctions sont dites \textbf{transcendantes} car elles << transcendent >> ou << dépassent >> l'algèbre.

Nous allons nous intéresser particulièrement à deux grandes familles importantes de fonctions transcendantes : les fonctions \emph{trigonométriques} et les fonctions \emph{exponentielles}. Ces fonctions apparaissent naturellement dans une multitude d'applications variées.

\subsection{Fonctions trigonométriques}

\subsubsection{Notions élémentaires de trigonométrie}

\subsubsection{Dérivées des fonctions trigonométriques}

\subsubsection{Fonctions trigonométriques inverses}

\subsubsection{Dérivées des fonctions trigonométriques inverses}

\begin{ex}
    Trouvons les extremums de $f(x) = 2x +10\arccot x$.
\end{ex}
\begin{ex}[Problème de maximisation de l'angle de Regiomontanus]
    Une personne désire admirer un tableau accroché sur un mur au-dessus du niveau de ses yeux. On veut trouver la distance au mur qui maximise l'angle de vision du tableau (voir la figure).
\end{ex}
\subsection{Fonctions exponentielles}

\subsubsection{Notions élémentaires de fonctions exponentielles et logarithmes}

\subsubsection{Dérivées des fonctions exponentielles et logarithmes}

\begin{ex}
    Calculer la dérivée des fonctions suivantes :
   \begin{multicols}{2}
        \begin{enumerate}
        \item $f(x) = 4^x - 5 \log_9 x$
        \item $g(x) = 3e^x + 10x^3 \ln x$
        \item $h(x) = e^{x^4 - 3x^2 + 9}$
        \item $i(x) = \ln(x^{-4} + x^4)$
    \end{enumerate}
    \end{multicols}
\end{ex}
\begin{ex}
    Soit la fonction $f(x) = e^{-\frac{x^2}{2}}$ (appelée \emph{fonction gaussienne} en l'honneur du mathématicien Carl Friedrich Gauss).
    \begin{enumerate}
        \item Faire une étude complète de $f$.
        \item Trouver le rectangle d'aire maximale inscrit entre le graphe de $f$ et l'axe des $x$.
    \end{enumerate}
\end{ex}

\subsubsection{Dérivation logarithmique}

Depuis quelque temps déjà, nous savons comment dériver toute fonction de la forme $f(x) = x^r$, où l'exposant $r \in \R$ est une constante. Dans la sous-section précédente, nous avons vu comment dériver toute fonction de la forme $f(x) = b^x$, où la base $b > 0$ est n'importe quelle constante strictement positive. Nous pouvons même dériver toute fonction de la forme $f(x) = b(x)^r$ ou bien $f(x) = b^{r(x)}$, à l'aide de la dérivation en chaîne. Mais qu'arrive-t-il si la base \emph{et} l'exposant dépendent simultanément de $x$? Autrement dit, comment peut-on calculer la dérivée d'une fonction ayant la forme $f(x) = b(x)^{r(x)}$?

L'idée derrière cette technique, appelée la \textbf{dérivation logarithmique}, est d'appliquer un logarithme à $f(x)$ avant d'effectuer la dérivée, dans le but d'abaisser l'exposant $r(x)$ devant le logarithme : \[ \ln(f(x)) = \ln(b(x)^{r(x)}) = r(x) \ln(b(x)). \] (On utilise le logarithme naturel parce qu'il est le plus simple à dériver.) En dérivant l'expression précédente (en chaîne à gauche, avec la règle du produit à droite), on obtient \[ \frac{f'(x)}{f(x)} = r'(x)\ln(b(x)) + r(x)\frac{b'(x)}{b(x)}, \] d'où l'expression un peu intimidante \[ f'(x) = f(x) \left(  r'(x)\ln(b(x)) + r(x)\frac{b'(x)}{b(x)} \right). \] En pratique, on ne se souvient jamais de la forme exacte de cette grosse expression, donc on refait plutôt le calcul à chaque fois. (Mais chapeau si vous arrivez à la retenir par cœur!)

\begin{rem}
    Il existe une approche alternative à cette méthode. Elle consiste à réécrire la fonction à base variable $f(x) = b(x)^{r(x)}$ sous la forme $f(x) = e^{\ln(b(x)^{r(x)})} = e^{r(x) \ln(b(x))}$. Ensuite, on dérive en chaîne pour obtenir $f'(x) = f(x)(r(x) \ln(b(x)))'$, ce qui donne la même expression que celle trouvée précédemment.
\end{rem}

\begin{ex}
    Calculer la dérivée de $f(x) = x^{\sin x}$ avec la dérivation logarithmique.
    
    On pourrait directement appliquer la formule ci-dessus, mais refaisons le calcul dans ce cas particulier. On a $\ln(f(x)) = \sin x \ln x$, dont la dérivée fait $\cos(x) \ln x + \frac{\sin x}{x}$. Ainsi, la dérivée recherchée est $f'(x) = x^{\sin x}\left( \cos x \ln x + \frac{\sin x}{x} \right)$.
\end{ex}
\end{document}
